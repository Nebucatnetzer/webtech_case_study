\documentclass[a4paper, 11pt]{article}
\include{style}
\bibliography{bib,andreas_general,andreas}
%\include{glossary}
\begin{document}

\begin{titlepage}
    \centering
    %\includegraphics[width=0.15\textwidth]{example-image-1x1}\par\vspace{1cm}
    {\scshape\LARGE IBZ-Schulen AG\par}
    %\vspace{0.5cm}
    {\scshape\Large Aarau\par}
    \vspace{1.5cm}
    {\huge\bfseries Casestudy Webtechnologien\par}
    \vspace{2cm}
    {\Large\itshape Hörler Ivan, Zweili Andreas, TI-AR-17W\par}
    \vfill

% Bottom of the page
    {\large \today\par}
\thispagestyle{empty}
\end{titlepage}


\newpage
\renewcommand{\abstractname}{Management Summary}
\begin{abstract}
Dies ist die Dokumentation für die zweite Case Study im Fach
Webtechnologien von Ivan Hörler und Andreas Zweili. Welche diese im
Rahmen ihres 5. Semesters an der IBZ Schule in Aarau erarbeiteten. Die
Case Study behandelt dabei das Erstellen eines Web-Shops und der dafür
gewählten Werkzeuge, die Projektplanung sowie die dabei aufgetretenen
Probleme.

Zusätzlich sollen auch die Erfahrungen der Studenten im Zusammenhang
des verwendeten Frameworks Django aufgezeigt werden. Dieses ist nicht
Teil des Kurikulums weshalb diese Arbeit interessante zusätzliche
Möglichkeiten im Bereich der Entwicklung von Webapplikationen
aufzeigen kann.
\end{abstract}

\newpage
\microtypesetup{protrusion=false} % disables protrusion locally in the document
\tableofcontents % prints Table of Contents
\microtypesetup{protrusion=true} % enables protrusion
\newpage


\include{doku}

\newpage
\nocite{*}
\printbibliography[heading=bibintoc]

\newpage
\microtypesetup{protrusion=false}
\listoffigures
\microtypesetup{protrusion=true}
\
\newpage

\microtypesetup{protrusion=false}
\listoftables
\microtypesetup{protrusion=true}
\
\newpage


%\printglossaries
\end{document}


%%% Local Variables:
%%% mode: latex
%%% TeX-master: t
%%% End:
